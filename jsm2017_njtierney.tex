\documentclass[]{article}
\usepackage{lmodern}
\usepackage{amssymb,amsmath}
\usepackage{ifxetex,ifluatex}
\usepackage{fixltx2e} % provides \textsubscript
\ifnum 0\ifxetex 1\fi\ifluatex 1\fi=0 % if pdftex
  \usepackage[T1]{fontenc}
  \usepackage[utf8]{inputenc}
\else % if luatex or xelatex
  \ifxetex
    \usepackage{mathspec}
  \else
    \usepackage{fontspec}
  \fi
  \defaultfontfeatures{Ligatures=TeX,Scale=MatchLowercase}
\fi
% use upquote if available, for straight quotes in verbatim environments
\IfFileExists{upquote.sty}{\usepackage{upquote}}{}
% use microtype if available
\IfFileExists{microtype.sty}{%
\usepackage{microtype}
\UseMicrotypeSet[protrusion]{basicmath} % disable protrusion for tt fonts
}{}
\usepackage[margin=1in]{geometry}
\usepackage{hyperref}
\hypersetup{unicode=true,
            pdftitle={Building on ggplot2 for exploration of missing values},
            pdfauthor={Nicholas Tierney},
            pdfborder={0 0 0},
            breaklinks=true}
\urlstyle{same}  % don't use monospace font for urls
\usepackage{color}
\usepackage{fancyvrb}
\newcommand{\VerbBar}{|}
\newcommand{\VERB}{\Verb[commandchars=\\\{\}]}
\DefineVerbatimEnvironment{Highlighting}{Verbatim}{commandchars=\\\{\}}
% Add ',fontsize=\small' for more characters per line
\usepackage{framed}
\definecolor{shadecolor}{RGB}{248,248,248}
\newenvironment{Shaded}{\begin{snugshade}}{\end{snugshade}}
\newcommand{\KeywordTok}[1]{\textcolor[rgb]{0.13,0.29,0.53}{\textbf{{#1}}}}
\newcommand{\DataTypeTok}[1]{\textcolor[rgb]{0.13,0.29,0.53}{{#1}}}
\newcommand{\DecValTok}[1]{\textcolor[rgb]{0.00,0.00,0.81}{{#1}}}
\newcommand{\BaseNTok}[1]{\textcolor[rgb]{0.00,0.00,0.81}{{#1}}}
\newcommand{\FloatTok}[1]{\textcolor[rgb]{0.00,0.00,0.81}{{#1}}}
\newcommand{\ConstantTok}[1]{\textcolor[rgb]{0.00,0.00,0.00}{{#1}}}
\newcommand{\CharTok}[1]{\textcolor[rgb]{0.31,0.60,0.02}{{#1}}}
\newcommand{\SpecialCharTok}[1]{\textcolor[rgb]{0.00,0.00,0.00}{{#1}}}
\newcommand{\StringTok}[1]{\textcolor[rgb]{0.31,0.60,0.02}{{#1}}}
\newcommand{\VerbatimStringTok}[1]{\textcolor[rgb]{0.31,0.60,0.02}{{#1}}}
\newcommand{\SpecialStringTok}[1]{\textcolor[rgb]{0.31,0.60,0.02}{{#1}}}
\newcommand{\ImportTok}[1]{{#1}}
\newcommand{\CommentTok}[1]{\textcolor[rgb]{0.56,0.35,0.01}{\textit{{#1}}}}
\newcommand{\DocumentationTok}[1]{\textcolor[rgb]{0.56,0.35,0.01}{\textbf{\textit{{#1}}}}}
\newcommand{\AnnotationTok}[1]{\textcolor[rgb]{0.56,0.35,0.01}{\textbf{\textit{{#1}}}}}
\newcommand{\CommentVarTok}[1]{\textcolor[rgb]{0.56,0.35,0.01}{\textbf{\textit{{#1}}}}}
\newcommand{\OtherTok}[1]{\textcolor[rgb]{0.56,0.35,0.01}{{#1}}}
\newcommand{\FunctionTok}[1]{\textcolor[rgb]{0.00,0.00,0.00}{{#1}}}
\newcommand{\VariableTok}[1]{\textcolor[rgb]{0.00,0.00,0.00}{{#1}}}
\newcommand{\ControlFlowTok}[1]{\textcolor[rgb]{0.13,0.29,0.53}{\textbf{{#1}}}}
\newcommand{\OperatorTok}[1]{\textcolor[rgb]{0.81,0.36,0.00}{\textbf{{#1}}}}
\newcommand{\BuiltInTok}[1]{{#1}}
\newcommand{\ExtensionTok}[1]{{#1}}
\newcommand{\PreprocessorTok}[1]{\textcolor[rgb]{0.56,0.35,0.01}{\textit{{#1}}}}
\newcommand{\AttributeTok}[1]{\textcolor[rgb]{0.77,0.63,0.00}{{#1}}}
\newcommand{\RegionMarkerTok}[1]{{#1}}
\newcommand{\InformationTok}[1]{\textcolor[rgb]{0.56,0.35,0.01}{\textbf{\textit{{#1}}}}}
\newcommand{\WarningTok}[1]{\textcolor[rgb]{0.56,0.35,0.01}{\textbf{\textit{{#1}}}}}
\newcommand{\AlertTok}[1]{\textcolor[rgb]{0.94,0.16,0.16}{{#1}}}
\newcommand{\ErrorTok}[1]{\textcolor[rgb]{0.64,0.00,0.00}{\textbf{{#1}}}}
\newcommand{\NormalTok}[1]{{#1}}
\usepackage{graphicx,grffile}
\makeatletter
\def\maxwidth{\ifdim\Gin@nat@width>\linewidth\linewidth\else\Gin@nat@width\fi}
\def\maxheight{\ifdim\Gin@nat@height>\textheight\textheight\else\Gin@nat@height\fi}
\makeatother
% Scale images if necessary, so that they will not overflow the page
% margins by default, and it is still possible to overwrite the defaults
% using explicit options in \includegraphics[width, height, ...]{}
\setkeys{Gin}{width=\maxwidth,height=\maxheight,keepaspectratio}
\IfFileExists{parskip.sty}{%
\usepackage{parskip}
}{% else
\setlength{\parindent}{0pt}
\setlength{\parskip}{6pt plus 2pt minus 1pt}
}
\setlength{\emergencystretch}{3em}  % prevent overfull lines
\providecommand{\tightlist}{%
  \setlength{\itemsep}{0pt}\setlength{\parskip}{0pt}}
\setcounter{secnumdepth}{0}
% Redefines (sub)paragraphs to behave more like sections
\ifx\paragraph\undefined\else
\let\oldparagraph\paragraph
\renewcommand{\paragraph}[1]{\oldparagraph{#1}\mbox{}}
\fi
\ifx\subparagraph\undefined\else
\let\oldsubparagraph\subparagraph
\renewcommand{\subparagraph}[1]{\oldsubparagraph{#1}\mbox{}}
\fi

%%% Use protect on footnotes to avoid problems with footnotes in titles
\let\rmarkdownfootnote\footnote%
\def\footnote{\protect\rmarkdownfootnote}

%%% Change title format to be more compact
\usepackage{titling}

% Create subtitle command for use in maketitle
\newcommand{\subtitle}[1]{
  \posttitle{
    \begin{center}\large#1\end{center}
    }
}

\setlength{\droptitle}{-2em}
  \title{Building on ggplot2 for exploration of missing values}
  \pretitle{\vspace{\droptitle}\centering\huge}
  \posttitle{\par}
  \author{Nicholas Tierney}
  \preauthor{\centering\large\emph}
  \postauthor{\par}
  \date{}
  \predate{}\postdate{}


\begin{document}
\maketitle

\section{Introduction}\label{introduction}

Missing data is ubiquitous in data analysis. ggplot2, an implementation
of the grammar of graphics is an incredibly popular way to produce data
visualisations does not currently support missing data (Wickham 2009).
Principles of tidy data (Wickham 2014) states that each row is an
observation and each column is a variable, which makes it easy and
consistent to perform data manipulation and wrangling. However, there
are currently no guidelines for representing additional missing data
structures in a tidy format. This paper describes approaches for
exploring missing data structure with minimal deviation from the common
workflows of ggplot and tidy data structures.

\section{Missing Data Mechanisms}\label{missing-data-mechanisms}

Canonical sources of missing data are questionnaires. Data obtained from
questionnaires are often subject to both unknown and known missingness
structure. Unknown missing data structure may arise from respondents
accidentally failing to answer questions or inadvertently providing
inappropriate answers. Known missing data structure data may arise due
to the structure of the questionnaire. For example, the first question
on a survey might be: `If YES, skip to question 4', resulting in
questions 2 and 3 missing. If the structure of the questionnaire is
known, this type of missingness can be evaluated easily. However, if
this information is not available, the mechanism responsible for
producing missing data must be inferred from the data. Longitudinal
studies are also sources of missing data, where participants may not
return for future testing sessions. In these cases it is difficult,
sometimes impossible, to ascertain the reason for the dropouts, and
hence, whether the missingness structure is known or unknown.

There are a two approaches to analysis of data with missing values,
deletion and imputation. Deletion methods drop variables or cases,
depending on the amount of missing data, and imputation methods replace
the missing values with some other value estimated from the data. It is
now widely regarded as best practice to impute these values, however in
order for estimates to be unbiased, it is essential to understand the
missingness structure and mechanisms (Little 1988; Rubin 1976; Simon and
Simonoff 1986; Schafer and Graham 2002).

\subsection{Existing packages for handling missing
data}\label{existing-packages-for-handling-missing-data}

Software focussing on missing data typically focus on imputation or
visualisation. Packages such as mice, mi, norm, and Amelia provide
functions to facilitate imputation, and use a wide range of methods,
from mean or median imputation, to regression or machine learning, to
Bayesian methodologies, as well as providing diagnostics on the
imputations themselves (Buuren and Groothuis-Oudshoorn 2011; Su et al.
2011; Schafer and Novo 2013; Honaker et al. 2011).

Missing data visualisation packages include the R package VIM, and the
stand alone softwares MANET, ggobi, MissingDataGUI, and to a more
limited extent, ggplot2 (Cheng et al. 2015; Unwin et al. 1996; Swayne et
al. 2003; Templ et al. 2011; Wickham 2009). MANET (Missings Are Now
Equally Treated), provides univariate visualisations of missing data
using linked brushing between a reference plot of the missingness for
each variable, and a plot of the data as a histogram or barplot. ggobi
extends the univariate linked brushing of MANET to multivariate, using
parallel co-ordinate plots. ggobi also provided incoporated missingness
into scatterplots, displaying missing values from one variable as 10\%
below the minimum value on the other axis. MissingDataGUI provides a
user interface for exploring missing data structure both numerically and
visually. Using a GUI to explore missing data makes it easier to glean
valuable insights into important structures, but may then make it hard
to incorporate these unscripted insights into reproducible analyses, and
may also distract and break the workflow from statistical analysis.

VIM (Visualising and Imputing Missing Data) is an R package that
provides methods for both imputation and visualisation of missing data.
In particular it provides visualisations that identify observed,
imputed, and missing values. VIM also identifies imputed cases by adding
a suffix to a variable, so Var1 would have a sibling indicator column,
Var1\_imp, where each case is TRUE or FALSE to indicate imputation.
Although VIM provides R functions to visualise and impute missing data,
it's syntax for data manipulation and visualisation is difficult to
extend, and does not follow tidy data principles. ggplot2 currently only
provides visualisation of missing values for categories treating
categories as NA values. For all other plots, ggplot2 prints a warning
message of the number of missing values ommited.

There are many ways to explore missing data structure and imputation,
however there is no unified methodology to explore, or visualise missing
data. We now discuss ways to represent missing data that fit in with the
grammar of graphics and tidy data.

\section{Data structures for missing
data}\label{data-structures-for-missing-data}

Representing missing data structure is achieed using the shadow matrix,
introduced in Swayne and Buja Swayne and Buja (1998; 1998). The shadow
matrix is the same dimension as the data, and consists of binary
indicators of missingness of data values, where missing is represented
as ``NA'', and not missing is represented as ``!NA''. Although these may
be represented as 1 and 0, respectively. This representation can be seen
in figure 1 below, adding the sufic ``\_NA" to the variables. This
structure can also be extended to allow for additional factor levels to
be created. For example 0 indicates data presence, 1 indicates missing
values, 2 indicates imputed value, and 3 might indicate a particular
type or class of missingness, where reasons for missingness might be
known or inferred. The data matrix can also be augmented to include the
shadow matrix, which facilitates visualisation of univariate and
bivariate missing data visualisations. Another format is to display it
in long form, which facilitates heatmap style visualisations. This
approach can be very helpful for giving an overview of which variables
contain the most missingness. Methods can also be applied to rearrange
rows and columns to find clusters, and identify other interesting
features of the data that may have previously been hidden or unclear.

\begin{figure}[h]
\centering
\includegraphics[width=270pt]{diagram.png}
\end{figure}

\textbf{Figure 1. Illustration of data structures for facilitating
visualisation of missings and not missings.}

\section{Visualising missing data}\label{visualising-missing-data}

\textbf{Heatmap}

A missing data heatmap is shown below using the \texttt{vis\_miss}
command from the \texttt{visdat} package. This displays the the
airquality dataset included in base R, which contains Daily air quality
measurements in New York, May to September 1973.

\includegraphics{jsm2017_njtierney_files/figure-latex/unnamed-chunk-1-1.png}

\textbf{Figure 2. Heatmaps of missing data (left) raw (right) ordered by
clustering on rows and columns.}

Similar approaches have been used in other missing data packages such as
VIM, mi, Amelia, and MissingDataGUI. However this plot is created in the
ggplot framework, giving users greater control over the plot appearance.
The user can also apply clustering of the rows and columns using the
\texttt{cluster\ =\ TRUE} argument (shown on the right).

\textbf{Univariate plots split by missingness}

An advantage of the augmented shadow format, where the data and shadow
are side by side, is that it allows for examining univariate
distributions according to the presence or absence of another variable.
The plot below shows the values of temperature when ozone is present and
missing, on the left is a faceted histogram, and on the right is an
overlayed density.

\begin{verbatim}
ggplot(data = bind_shadow(airquality),
       aes(x = Temp)) + 
  geom_histogram() + 
  facet_wrap(~Ozone_NA,
             ncol = 1)

ggplot(data = bind_shadow(airquality),
       aes(x = Temp,
           colour = Ozone_NA)) + 
  geom_density()
\end{verbatim}

\includegraphics{jsm2017_njtierney_files/figure-latex/bind-shadow-density-1.png}

\textbf{Figure 3., Two representations of temperature showing
missingness on ozone (left) facetted, (right) coloured. Main difference
in the distribution is that there is a cluster of low temperature
observations with missing ozone values.}

Using this data structure allows for the user to directly refer to the
variable for which they want to explore the effect of missingness using
the suffix \_NA after the variable. In the case above, the user is
looking at a histogram of temperature, but is then able to look at how
many temperature values are affected by missingness of ozone. In cases
where there is no missing data in the variable that they want to
``split'' the missingness by, the plot simple returns a single facetted
plot.

Another method of visualisation can be explored using
\texttt{geom\_missing\_point()} from the \texttt{ggmissing} package:

\begin{Shaded}
\begin{Highlighting}[]
\NormalTok{p1 <-}\StringTok{ }\KeywordTok{ggplot}\NormalTok{(}\DataTypeTok{data =} \NormalTok{airquality,}
       \KeywordTok{aes}\NormalTok{(}\DataTypeTok{x =} \NormalTok{Ozone,}
           \DataTypeTok{y =} \NormalTok{Solar.R)) +}\StringTok{ }
\StringTok{  }\KeywordTok{geom_missing_point}\NormalTok{() +}\StringTok{ }
\StringTok{  }\KeywordTok{theme}\NormalTok{(}\DataTypeTok{aspect.ratio =} \DecValTok{1}\NormalTok{)}

\NormalTok{p2 <-}\StringTok{ }\KeywordTok{ggplot}\NormalTok{(}\DataTypeTok{data =} \NormalTok{airquality,}
       \KeywordTok{aes}\NormalTok{(}\DataTypeTok{x =} \NormalTok{Temp,}
           \DataTypeTok{y =} \NormalTok{Ozone)) +}\StringTok{ }
\StringTok{  }\KeywordTok{geom_missing_point}\NormalTok{() +}\StringTok{ }
\StringTok{  }\KeywordTok{theme}\NormalTok{(}\DataTypeTok{aspect.ratio =} \DecValTok{1}\NormalTok{)}

\NormalTok{gridExtra::}\KeywordTok{grid.arrange}\NormalTok{(p1,p2,}\DataTypeTok{ncol=}\DecValTok{2}\NormalTok{)}
\end{Highlighting}
\end{Shaded}

\includegraphics{jsm2017_njtierney_files/figure-latex/ggeom_missing-1.png}

\textbf{Figure 4., Scatterplots with missings displayed at 10\% below
(left) ozone versus and solar radiation (right) ozone versus
temperature.}

This replaces missing values to be 10\% below the minimum value, a
technique borrowed from ggobi. The missing values are also different
colours to make missingness preattentive (Treisman 1985). In this plot
we see that there is a mostly uniform spread of missing values for
Solar.R and Ozone. As \texttt{geom\_missing\_point} is a defined
geometry for ggplot2, it allows users to have full customisation as they
normally would with ggplot.

\section{Numerical Summaries for missing
data}\label{numerical-summaries-for-missing-data}

Numerical summaries of missing data are also made easy with some helper
functions from the \texttt{ggmissing} package, which provides tidy
functions that return either single numbers or dataframes. The
\texttt{percent\_missing\_*} functions help find the proportion of
missing values in the data overall, in cases, or in variables.

\begin{Shaded}
\begin{Highlighting}[]
\CommentTok{# Proportion elements in dataset that contains missing values}
\KeywordTok{percent_missing_df}\NormalTok{(airquality)}
\end{Highlighting}
\end{Shaded}

\begin{verbatim}
## [1] 4.793028
\end{verbatim}

\begin{Shaded}
\begin{Highlighting}[]
\CommentTok{# Proportion of variables that contain any missing values}
\KeywordTok{percent_missing_var}\NormalTok{(airquality)}
\end{Highlighting}
\end{Shaded}

\begin{verbatim}
## [1] 33.33333
\end{verbatim}

\begin{Shaded}
\begin{Highlighting}[]
 \CommentTok{# Proportion of cases that contain any missing values}
\KeywordTok{percent_missing_case}\NormalTok{(airquality)}
\end{Highlighting}
\end{Shaded}

\begin{verbatim}
## [1] 27.45098
\end{verbatim}

We can also look at the number and percent of missings in each case and
variable with \texttt{summary\_missing\_case}, and
\texttt{summary\_missing\_var}.

\begin{Shaded}
\begin{Highlighting}[]
\KeywordTok{summary_missing_case}\NormalTok{(airquality) %>%}\StringTok{ }\KeywordTok{slice}\NormalTok{(}\DecValTok{1}\NormalTok{:}\DecValTok{5}\NormalTok{)}
\end{Highlighting}
\end{Shaded}

\begin{verbatim}
## # A tibble: 5 × 3
##    case n_missing  percent
##   <int>     <int>    <dbl>
## 1     1         0  0.00000
## 2     2         0  0.00000
## 3     3         0  0.00000
## 4     4         0  0.00000
## 5     5         2 33.33333
\end{verbatim}

\begin{Shaded}
\begin{Highlighting}[]
\KeywordTok{summary_missing_var}\NormalTok{(airquality)}
\end{Highlighting}
\end{Shaded}

\begin{verbatim}
## # A tibble: 6 × 3
##   variable n_missing   percent
##      <chr>     <int>     <dbl>
## 1    Ozone        37 24.183007
## 2  Solar.R         7  4.575163
## 3     Wind         0  0.000000
## 4     Temp         0  0.000000
## 5    Month         0  0.000000
## 6      Day         0  0.000000
\end{verbatim}

Tabulations of the number of missings in each case or variable can be
calculated with \texttt{table\_missing\_case} and
\texttt{table\_missing\_var}.

\begin{Shaded}
\begin{Highlighting}[]
\KeywordTok{table_missing_case}\NormalTok{(airquality)}
\end{Highlighting}
\end{Shaded}

\begin{verbatim}
## # A tibble: 3 × 3
##   n_missing_in_case n_cases  percent
##               <int>   <int>    <dbl>
## 1                 0     111 72.54902
## 2                 1      40 26.14379
## 3                 2       2  1.30719
\end{verbatim}

\begin{Shaded}
\begin{Highlighting}[]
\KeywordTok{table_missing_var}\NormalTok{(airquality)}
\end{Highlighting}
\end{Shaded}

\begin{verbatim}
## # A tibble: 3 × 3
##   n_missing_in_var n_vars  percent
##              <int>  <int>    <dbl>
## 1                0      4 66.66667
## 2                7      1 16.66667
## 3               37      1 16.66667
\end{verbatim}

\section{Discussion}\label{discussion}

In this paper we discussed missing data mechanisms, existing packages
for imputation and visualisation of missing data, and the limitations of
current missing data exploration and visualisaion softwares. We then
discussed data structures for missing data, and showed how these can be
used following tidy data principles, and how to effectively present
visualisations and numerical summaries using the R packages ggmissing
and visdat, available for download on github:
\url{https://github.com/njtierney/ggmissing}, and
\url{https://github.com/njtierney/visdat}.

It is worthwhile to note the trade off between storage and computation
of the augmented shadow matrix. When storage of data is an issue, it may
not be practial to bind the shadow matrix to the regular data. Instead,
it may be more effective to perform the computation for the column of
interest when necessary. However, the shadow matrix can also allow for
more complex types of missingness to be expressed, and so there are
additional benefits to storing data in this way. For example, missing,
\texttt{NA},and not missing, \texttt{!NA}, could be extended to describe
different mechanisms for missingness, e.g., \texttt{NA\_mechanism\_A},
and \texttt{NA\_mechanism\_B}, or even imputed values
\texttt{value\_imputed}. These could then be combined with the same
sorts of plots and numerical summaries to provide diagnostics.

Future research should focus on developing techniques for identifying
missingness mechanisms and methods for encoding mechanisms into the
shadow matrix. Further work could also be done on developing methods to
store single and multiple imputations into the shadow matrix, and
methods to visualise these imputations using ggplot geoms, and assess
them with numerical summaries.

\section*{References}\label{references}
\addcontentsline{toc}{section}{References}

\hypertarget{refs}{}
\hypertarget{ref-mice}{}
Buuren, Stef van, and Karin Groothuis-Oudshoorn. 2011. ``Mice:
Multivariate Imputation by Chained Equations in R.'' \emph{J. Stat.
Softw.} 45 (1): 1--67.

\hypertarget{ref-cheng2015}{}
Cheng, Xiaoyue, Dianne Cook, Heike Hofmann, and others. 2015. ``Visually
Exploring Missing Values in Multivariable Data Using a Graphical User
Interface.'' \emph{Journal of Statistical Software} 68 (1). Foundation
for Open Access Statistics: 1--23.

\hypertarget{ref-Amelia}{}
Honaker, James, Gary King, Matthew Blackwell, and Others. 2011. ``Amelia
II: A Program for Missing Data.'' \emph{J. Stat. Softw.} 45 (7): 1--47.

\hypertarget{ref-Little1988}{}
Little, Roderick JA. 1988. ``A Test of Missing Completely at Random for
Multivariate Data with Missing Values.'' \emph{Journal of the American
Statistical Association} 83 (404). Taylor \& Francis: 1198--1202.

\hypertarget{ref-Rubin1976}{}
Rubin, Donald B. 1976. ``Inference and Missing Data.'' \emph{Biometrika}
63 (3). Biometrika Trust: 581--92.

\hypertarget{ref-Schafer2002}{}
Schafer, Joseph L., and John W. Graham. 2002. ``Missing data: Our view
of the state of the art.'' \emph{Psychological Methods} 7 (2): 147--77.
doi:\href{https://doi.org/10.1037//1082-989X.7.2.147}{10.1037//1082-989X.7.2.147}.

\hypertarget{ref-norm}{}
Schafer, Joseph L., and Alvaro A. Novo. 2013. \emph{Norm: Analysis of
Multivariate Normal Datasets with Missing Values}.
\url{https://CRAN.R-project.org/package=norm}.

\hypertarget{ref-simon-simonoff}{}
Simon, Gary A., and Jeffrey S Simonoff. 1986. ``Diagnostic Plots for
Missing Data in Least Squares Regression.'' \emph{Journal of the
American Statistical Association} 81 (394). Taylor \& Francis Group:
501--9.

\hypertarget{ref-mi}{}
Su, Yu-Sung, Andrew Gelman, Jennifer Hill, and Masanao Yajima. 2011.
``Multiple Imputation with Diagnostics (Mi) in R: Opening Windows into
the Black Box.'' \emph{J. Stat. Softw.} 45 (1): 1--31.

\hypertarget{ref-Swayne1998}{}
Swayne, Deborah F, and Andreas Buja. 1998. ``Missing Data in Interactive
High-Dimensional Data Visualization.'' \emph{Computational Statistics}
13 (1). Citeseer: 15--26.

\hypertarget{ref-swayne2003ggobi}{}
Swayne, Deborah F, Duncan Temple Lang, Andreas Buja, and Dianne Cook.
2003. ``GGobi: Evolving from Xgobi into an Extensible Framework for
Interactive Data Visualization.'' \emph{Computational Statistics \& Data
Analysis} 43 (4). Elsevier: 423--44.

\hypertarget{ref-vim}{}
Templ, Matthias, Andreas Alfons, Alexander Kowarik, and Bernd Prantner.
2011. ``VIM: Visualization and Imputation of Missing Values.'' \emph{R
Package Version} 2 (3).

\hypertarget{ref-treisman1985}{}
Treisman, Anne. 1985. ``Preattentive Processing in Vision.''
\emph{Computer Vision, Graphics, and Image Processing} 31 (2). Elsevier:
156--77.

\hypertarget{ref-Unwin1996}{}
Unwin, Antony, George Hawkins, Heike Hofmann, and Bernd Siegl. 1996.
``Interactive Graphics for Data Sets with Missing Values - Manet.''
\emph{Journal of Computational and Graphical Statistics} 5 (2). Taylor
\& Francis Group: 113--22.

\hypertarget{ref-wickham2009ggplot2}{}
Wickham, Hadley. 2009. \emph{Ggplot2: Elegant Graphics for Data
Analysis}. Springer Science \& Business Media.

\hypertarget{ref-wickham2014}{}
---------. 2014. ``Tidy Data.'' \emph{Journal of Statistical Software}
59 (10).


\end{document}
